%% RiSE Latex Template - version 0.5
%%
%% RiSE's latex template for thesis and dissertations
%% http://risetemplate.sourceforge.net
%%
%% (c) 2012 Yguaratã Cerqueira Cavalcanti (yguarata@gmail.com)
%%          Vinicius Cardoso Garcia (vinicius.garcia@gmail.com)
%%
%% This document was initially based on UFPEThesis template, from Paulo Gustavo
%% S. Fonseca.
%%
%% ACKNOWLEDGEMENTS
%%
%% We would like to thanks the RiSE's researchers community, the 
%% students from Federal University of Pernambuco, and other users that have
%% been contributing to this projects with comments and patches.
%%
%% GENERAL INSTRUCTIONS
%%
%% We strongly recommend you to compile your documents using pdflatex command.
%% It is also recommend use the texlipse plugin for Eclipse to edit your documents.
%%
%% Options for \documentclass command:
%%         * Idiom
%%           pt   - Portguese (default)
%%           en   - English
%%
%%         * Text type
%%           bsc  - B.Sc. Thesis
%%           msc  - M.Sc. Thesis (default)
%%           qual - PHD qualification (not tested yet)
%%           prop - PHD proposal (not tested yet)
%%           phd  - PHD thesis
%%
%%         * Media
%%           scr  - to eletronic version (PDF) / see the users guide
%%
%%         * Pagination
%%           oneside - unique face press
%%           twoside - two faces press
%%
%%		   * Line spacing
%%           singlespacing  - the same as using \linespread{1}
%%           onehalfspacing - the same as using \linespread{1.3}
%%           doublespacing  - the same as using \linespread{1.6}
%%
%% Reference commands. Use the following commands to make references in your
%% text:
%%          \figref  -- for Figure reference
%%          \tabref  -- for Table reference
%%          \eqnref  -- for equation reference
%%          \chapref -- for chapter reference
%%          \secref  -- for section reference
%%          \appref  -- for appendix reference
%%          \axiref  -- for axiom reference
%%          \conjref -- for conjecture reference
%%          \defref  -- for definition reference
%%          \lemref  -- for lemma reference
%%          \theoref -- for theorem reference
%%          \corref  -- for corollary reference
%%          \propref -- for proprosition reference
%%          \pgref   -- for page reference
%%
%%          Example: See \chapref{chap:introduction}. It will produce 
%%                   'See Chapter 1', in case of English language.

\documentclass[pt,twoside,onehalfspacing,msc]{risethesis}

\usepackage[english]{babel}
\usepackage{colortbl}
\usepackage{color}
\usepackage[table]{xcolor}
\usepackage{microtype}
\usepackage{bibentry}
\usepackage{subfigure}
\usepackage{multirow}
\usepackage{rotating}
\usepackage{booktabs}
\usepackage{pdfpages}
\usepackage{caption}
\usepackage{lipsum}

\captionsetup[table]{position=top,justification=centering,width=.85\textwidth,labelfont=bf,font=small}
\captionsetup[lstlisting]{position=top,justification=centering,width=.85\textwidth,labelfont=bf,font=small}
\captionsetup[figure]{position=bottom,justification=centering,width=.85\textwidth,labelfont=bf,font=small}

%% Change the following pdf author attribute name to your name.
\usepackage[linkcolor=black,
            citecolor=blue,
            urlcolor=black,
            colorlinks,
            pdfpagelabels,
            pdftitle={Otimização de Portfólio usando Extreme Learning Machine},
            pdfauthor={Augusto César Benvenuto de Almeida}]{hyperref}

\address{RECIFE}

\universitypt{Universidade Federal de Pernambuco}
\universityen{Federal University of Pernambuco}

\departmentpt{Centro de Informática}
\departmenten{Center for Informatics}

\programpt{Pós-graduação em Ciência da Computação}
\programen{Graduate in Computer Science}

\majorfieldpt{Ciência da Computação}
\majorfielden{Computer Science}

\title{Otimização de Portfólio usando Extreme Learning Machine}

\date{2015}

\author{Augusto César Benvenuto de Almeida}
\adviser{Adriano Lorena Inácio de Oliveira}
\coadviser{???}

% Macros (defines your own macros here, if needed)
\def\x{\checkmark}

\begin{document}

\frontmatter

\frontpage

\presentationpage

\begin{fichacatalografica}
	\FakeFichaCatalografica % Comment this line when you have the correct file
%     \includepdf{fig_ficha_catalografica.pdf} % Uncomment this
\end{fichacatalografica}

\banca

\begin{dedicatory}
I dedicate this thesis to all my family, friends and professors who gave me the
necessary support to get here.
\end{dedicatory}

\acknowledgements
\lipsum[1-4]

\begin{epigraph}[]{Poul Anderson}
I have yet to see any problem, however complicated, which, when looked at in the
right way, did not become still more complicated.
\end{epigraph}

\resumo
% Escreva seu resumo no arquivo resumo.tex

\textcolor{red}{TODO}
\begin{itemize}
\color{red}
\item Firts item
\item Second item
\end{itemize}

A implementação da arquitetura segue uma estratégia de automação, também
elabo\-rada nesse trabalho, que possui dois componentes principais: um sistema
especialista baseado em regras (SEBR); e um modelo de recuperação de informação
(MRI) com técnicas de aprendizagem. Em nossa estratégia, esses dois componentes
são executados alternadamente em momentos diferentes a fim de atribuir uma SM
automaticamente. O SEBR processa regras simples e complexas, considerando
informações contextuais do projeto de software e da organização que o
desenvolve. O MRI é utilizado para fazer o casamento entre SMs e desenvolvedores
de acordo com o histórico de atribuições.

\begin{keywords}
Engenharia de Software, Manutenção e Evolução de Software, Gerenciamento de
Solicitações de Mudança, Atribuição Automática de Solicitações de Mudança
\end{keywords}

\abstract
% Write your abstract in a file called abstract.tex
The efficient management of \acp{cr} is fundamental for successful software
maintenance; however the assignment of \acp{cr} to developers is an expensive
aspects in this regard, due to the time and expertise demanded. To overcome
this, researchers have proposed automated approaches for \ac{cr} assignment.
Although these proposals present advances to this topic, they do not consider
many factors inherent to the assignments. Indeed, different complex factors may
have influence on \ac{cr} assignment, and most of them vary from one
organization to another. For instance, developers' workload, \acp{cr} severity,
interpersonal relationships, or developers know-how must be considered in the
assignments. Actually, as we demonstrate in this work, \ac{cr} assignment is a
complex activity and automated approaches cannot rely on simplistic solutions.
Ideally, it is necessary to consider and reason over contextual information in
order to provide an effective automation.

In this regarding, this work proposes, implements, and validates a context-aware
architecture to automate \ac{cr} assignment. The architecture emphasizes the
need for considering the different information available at the organization to
provide a more context-aware solution to automated \ac{cr} assignment. The
development of such architecture is supported by evidence synthesized from two
empirical studies: a survey with practitioners and a systematic mapping study.
The survey provided us with a set of requirements that automated approaches
should satisfy. In the mapping study, in turn, we figured out how
state-of-the-art approaches are implemented in regarding to these requirements,
concluding that many of them are not satisfied. In addition, new requirements
were identified in this mapping study.

For the implementation of the proposed architecture, we developed a strategy to
automate \ac{cr} assignments which is based on two main components: a \acf{rbes}
and an \acf{ir} model. The strategy coordinately applies these two components in
different steps to find the potential developer to a \ac{cr}. The \ac{rbes}
takes care of the simple and complex rules necessary to consider contextual
information in the assignments, e.g., to prevent assigning a \ac{cr} to a busy
or unavailable developer. Since these rules vary from one organization/project
to another, the \ac{rbes} facilitates their modification for different contexts.
On the other hand, the \ac{ir} model is useful to make use of the historical
information of \ac{cr} assignments to match \acp{cr} and developers.

\begin{keywords}
Software Engineering, Software Maintenance and Evolution, Change Request
Management, Automatic Change Request Assignment
\end{keywords}

% List of figures
\listoffigures

% List of tables
\listoftables

% List of acronyms
% Acronyms manual: http://linorg.usp.br/CTAN/macros/latex/contrib/acronym/acronym.pdf
\listofacronyms
\begin{acronym}[ACRONYM] 
% Change the word ACRONYM above to change the acronym column width.
% The column width is equals to the width of the word that you put.
% Read the manual about acronym package for more examples:
%   http://linorg.usp.br/CTAN/macros/latex/contrib/acronym/acronym.pdf
\acro{afm}[AFM]{Alphabet Frequency Matrix}
\acro{api}[API]{Application Programming Interface}
\acro{arima}[ARIMA]{Auto-Regressive Integrated Moving Average}
\acro{brn}[BRN]{Bug Report Network}
\acro{bts}[BTS]{Bug Triage System}
\acro{cas}[CAS]{Context-Aware Systems}
\acro{ccb}[CCB]{Change Control Board}
\acro{cr}[CR]{Change Request}
\acro{cvs}[CVS]{Concurrent Version System}
\acro{es}[ES]{Expert System}
\acro{floss}[FLOSS]{Free/Libre Open Source Software}
\acro{glr}[GLR]{Generalized Linear Regression}
\acro{gqm}[GQM]{Goal Question Metric}
\acro{html}[HTML]{HyperText Markup Language}
\acro{ir}[IR]{Information Retrieval}
\acro{irt}[IRT]{Recôncavo Institute of Technology}
\acro{jdt}[JDT]{Jazz Duplicate Finder}
\acro{lda}[LDA]{Latent Dirichlet Allocation}
\acro{loc}[LOC]{Lines of Code}
\acro{lsi}[LSI]{Latent Semantic Indexing}
\acro{ms}[MS]{Mapping Study}
\acro{msr}[MSR]{Mining Software Repositories}
\acro{nlp}[NLP]{Natural Language Processing}
\acro{promise}[PROMISE]{Predictive Models in Software Engineering}
\acro{rbes}[RBES]{Rule-Based Expert System}
\acro{rhel}[RHEL]{RedHat Enterprise Linux}
\acro{saas}[SaaS]{Software as a Service}
\acro{scm}[SCM]{Software Configuration Management}
\acro{serpro}[SERPRO]{Brazilian Federal Organization for Data Processing}
\acro{slr}[SLR]{Stepwise Linear Regression}
\acro{slr}[SLR]{Systematic Literature Review}
\acro{svd}[SVD]{Singular Value Decomposition}
\acro{svm}[SVM]{Support Vector Machine}
\acro{svn}[SVN]{Subversion}
\acro{tfidf}[TF-IDF]{Term Frequency-Inverse Document Frequency}
\acro{vsm}[VSM]{Vector Space Model}
\acro{xp}[XP]{Extreming Programming}
\end{acronym}

% Summary (tables of contents)
\tableofcontents

\mainmatter

\chapter{Introdução}
\label{chp:introdução}


Software maintenance starts as early as the first software artifacts are
delivered, and is characterized by its high cost and slow speed of
implementation~\citep{swebok2004}. It has been stated that it is the most
expensive activity of software development, taking up to 90\% of the total
costs~\citep{Eastwood1993,Erlikh2000}. However, despite of the high cost, it is
mandatory to ensure the success of the software project. \citet{Lehman1980}
argues, in his \emph{Continuing Change} law of software evolution, that the
modification of software is a fact of life for software systems if they are
intended to remain useful. \citet{Bennett2000} reinforced such an argument for
the specific case of useful and successful software, where almost all of them
have a common practice of stimulating user-generated \ac{cr}. Actually, software
maintenance is driven by \acp{cr} reported by many stakeholders, such as
developers, testers, team leaders, managers, and clients.


\lipsum[2-4]

\section{Problem Statement}
\label{sec:intro-problem-statement}

\lipsum[3-5]

\begin{enumerate}
  \item Firstly, the approaches available in the literature were designed to
  perform autonomously. That is, the software analysts do not have the control
  of the approach; they cannot modify the approach's behavior. Without
  such control, in turn, the approach cannot be properly calibrated. As a
  consequence, if the approach's performance is not satisfactory, it is simply
  discarded.
  \item Secondly, the reported values for accuracy of these approaches are
  still low. With low accuracy, the previous reason takes place. That is, as the
  approaches perform with low accuracy, and the software analysts do not have
  control over them, the approaches are simply discarded.
  \item Finally, the third reason concerns the lack of contextual information in
  those approaches. As is well known, software development companies are
  dynamic: developers move from projects; developers are hired/fired;
  developers enter in vacation or take a day off; and developers have different
  experiences. This dynamic influences the assignment of \acp{cr}. Thus,
  contextual information is a necessity in automated approaches.
\end{enumerate}

\section{Overview of the Proposal}
\lipsum[1-5]
\chapter{Fundamentação Teórica}

\section{Portfolio Optimization}

\lipsum[1-4]

\subsection{Subsection}

\lipsum[2-4]
\chapter{Revisão Bibliográfica}

\section{Portfolio Optimization}

\lipsum[2-4]
\chapter{Desenvolvimento}

\section{Introduction}

\lipsum[1-4]

\subsection{Subsection}

\lipsum[2-4]
\chapter{Conclusão}

\section{Introduction}

\lipsum[1-4]

\subsection{Subsection}

\lipsum[2-4]

% References

\begin{references}
  \bibliography{references}
\end{references}

% Appendix

\theappendix
\chapter{Mapping Study's Instruments}
\label{ap:mapping-study}

\begin{table}[!htp]
	\centering
	\caption{List of conferences on which the searches were performed.}
	\label{tbl:conferences_list}
	\rowcolors{2}{lightgray!30}{white}
	\resizebox{\columnwidth}{!}{
	\begin{tabular}{ll}
	\toprule
	\textbf{Acronym} & \textbf{Conference} \\
	\toprule
	APSEC & Asia Pacific Software Engineering Conference \\
	ASE   & IEEE/ACM International Conference on Automated Software Engineering \\
	CSMR  & European Conference on Software Maintenance and Reengineering \\
	ESEC  & European Software Engineering Conference \\
	ESEM  & International Symposium on Empirical Software Management and Measurement \\
	ICSE  & International Conference on Software Engineering \\
	ICSM  & International Conference on Software Maintenance \\
	ICST & International Conference on Software Testing \\
	InfoVis & IEEE Information Visualization Conference \\
	KDD   & ACM SIGKDD International Conference on Knowledge Discovery and Data Mining \\
	MSR   & Working Conference on Mining Software Repositories \\
	OOPSLA & Object-Oriented Programming, Systems, Languages and Applications \\
	QSIC  & International Conference On Quality Software \\
	SAC & ACM Symposium on Applied Computing \\
	SEAA & EUROMICRO Conference on Software Engineering and Advanced Applications\\
	SEDE & 19th International Conference on Software Engineering and Data Engineering \\
	SEKE  & International Conference on Software Engineering and Knowledge Engineering \\
	\bottomrule
	\end{tabular}
	}
\end{table}

\begin{table}[htp]
	\caption{List of journals in which the searches were performed.}
	\label{tbl:journals_list}
	\centering
	\rowcolors{2}{lightgray!30}{white}
	\begin{tabular}{l}
	\toprule
	\textbf{Journal title} \\
	\toprule
	ACM Transactions on Software Engineering and Methodology \\
	Automated Software Engineering \\
	Elsevier Information and Software Technology \\
	Elsevier Journal of Systems and Software \\
	Empirical Software Engineering \\
	IEEE Software \\
	IEEE Computer \\
	IEEE Transactions on Software Engineering \\
	International Journal of Software Engineering and Knowledge Engineering \\
	Journal of Software: Evolution and Process \\
	Software Quality Journal \\
	Journal of Software \\
	Software Practice and Experience Journal \\
	\bottomrule
	\end{tabular}
\end{table}

\begin{table}[h]
\centering
\footnotesize
 \rowcolors{2}{lightgray!30}{white}
\caption{Search string per Search Engine.}
\label{tbl:stringengine}
\begin{tabular}{p{.15\textwidth}p{.8\textwidth}}
\toprule
\textbf{Search Engine} & \textbf{Search String}\\
\toprule
   	 	Google Scholar &  bug report OR track OR triage ``change
   	 	request'' issue track OR request OR software OR ``modification request'' OR
   	 	``defect track'' OR ``software issue''  repositories maintenance evolution\\

   	 	ACM Portal & Abstract: "bug report" or Abstract:"change request"
   	 	or Abstract:"bug track" or Abstract:"issue track" or  Abstract:"defect
   	 	track" or Abstract:"bug triage" or Abstract: "software issue" or Abstract: "issue request"
   	 	or Abstract: "modification request") and  (Abstract:software or
   	 	Abstract:maintenance or Abstract:repositories or Abstract:repository \\

   	 	IEEExplorer (1) & (((((((((("Abstract": "bug report") OR
   	 	"Abstract":"change request") OR "Abstract":"bug track") OR "Abstract":"software issue") OR "Abstract":"issue request") OR
        "Abstract":"modification request") OR "Abstract":"issue track") OR
	    "Abstract":"defect track") OR "Abstract":"bug triage") AND
	    "Abstract":software)\\

         IEEExplorer (2) & (((((((((("Abstract": "bug report") OR
         "Abstract":"change request") OR "Abstract":"bug track") OR "Abstract":"software issue") OR
         "Abstract":"issue request") OR "Abstract":"modification request") OR
         "Abstract":"issue track") OR "Abstract":"defect track") OR
         "Abstract":"bug triage") AND "Abstract":maintenance)\\

         IEEExplorer (3) & (((((((((("Abstract": "bug report") OR
         "Abstract":"change request") OR "Abstract":"bug track") OR "Abstract":"software issue") OR
         "Abstract":"issue request") OR "Abstract":"modification request") OR
         "Abstract":"issue track") OR "Abstract":"defect track") OR
         "Abstract":"bug triage") AND "Abstract":repositories)\\

         IEEExplorer & (((((((((("Abstract": "bug report") OR
         "Abstract":"change request") OR "Abstract":"bug track") OR "Abstract":"software issue") OR
         "Abstract":"issue request") OR "Abstract":"modification request") OR
         "Abstract":"issue track") OR "Abstract":"defect track") OR
         "Abstract":"bug triage") AND "Abstract": repository)\\

         Citeseer Library & (abstract: "bug report" OR abstract:"change request" OR abstract:"bug track" OR abstract:"issue track" OR
	     abstract:"defect track" OR abstract:"bug triage" OR abstract: "software
	     issue" OR abstract: "issue request" OR abstract: "modification request")
	     AND (abstract:software OR abstract:maintenance OR abstract:repositories OR
	     abstract:repository)\\

	     Elsevier & ("bug report" OR "change
	     request" OR "bug track" OR "issue track" OR "defect track" OR "bug triage" OR "software issue" OR  "issue request" OR
	    "modification request") AND (software OR maintenance OR repositories OR
	    repository)\\

	    Scirus & ("bug report" OR "change request" OR "bug track" OR "issue track" OR  "defect track" OR "bug triage" OR
        "software issue" OR  "issue request" OR "modification request") AND
	    (software maintenance OR repositories OR repository) ANDNOT (medical OR
	    aerospace)\\

	    ScienceDirect & ("bug report" OR "change request" OR "bug track"
	     OR "issue track" OR "defect track" OR "bug triage" OR "issue request" OR
	     "modification request") AND LIMIT-TO(topics, "soft ware")\\

	     Scopus & ("bug report" OR "change request" OR "bug track" OR
	     "issue track" OR  "defect track" OR "bug triage" OR "software issue" OR
	     "issue request" OR "modification request") AND (software maintenance OR
	     repositories OR repository)\\

	     Wiley & ("bug report" OR "change request"
	     OR "bug track" OR "issue track" OR  "defect track" OR "bug triage" OR
         "software issue" OR  "issue request" OR "modification request") AND
	     (software maintenance OR repositories OR repository)\\

	     ISI Web\newline of Knowledge & ("bug report" OR "change request" OR "bug
	     track" OR "issue track" OR  "defect track" OR "bug triage" OR "software issue" OR  "issue request" OR "modification request") AND
	    (software maintenance OR repositories OR repository) ANDNOT (medical OR
	    aerospace)\\

	    SpringerLink & ("bug report" OR "change request" OR "bug track" OR "issue track" OR  "defect track" OR "bug triage" OR
        "software issue" OR  "issue request" OR "modification request") AND
	    (software maintenance OR repositories OR repository) ANDNOT (medical OR
	    aerospace)\\
	\bottomrule
\end{tabular}
\end{table}

\end{document}
