\chapter{Introdução}
\label{chp:introdução}


Software maintenance starts as early as the first software artifacts are
delivered, and is characterized by its high cost and slow speed of
implementation~\citep{swebok2004}. It has been stated that it is the most
expensive activity of software development, taking up to 90\% of the total
costs~\citep{Eastwood1993,Erlikh2000}. However, despite of the high cost, it is
mandatory to ensure the success of the software project. \citet{Lehman1980}
argues, in his \emph{Continuing Change} law of software evolution, that the
modification of software is a fact of life for software systems if they are
intended to remain useful. \citet{Bennett2000} reinforced such an argument for
the specific case of useful and successful software, where almost all of them
have a common practice of stimulating user-generated \ac{cr}. Actually, software
maintenance is driven by \acp{cr} reported by many stakeholders, such as
developers, testers, team leaders, managers, and clients.


\lipsum[2-4]

\section{Problem Statement}
\label{sec:intro-problem-statement}

\lipsum[3-5]

\begin{enumerate}
  \item Firstly, the approaches available in the literature were designed to
  perform autonomously. That is, the software analysts do not have the control
  of the approach; they cannot modify the approach's behavior. Without
  such control, in turn, the approach cannot be properly calibrated. As a
  consequence, if the approach's performance is not satisfactory, it is simply
  discarded.
  \item Secondly, the reported values for accuracy of these approaches are
  still low. With low accuracy, the previous reason takes place. That is, as the
  approaches perform with low accuracy, and the software analysts do not have
  control over them, the approaches are simply discarded.
  \item Finally, the third reason concerns the lack of contextual information in
  those approaches. As is well known, software development companies are
  dynamic: developers move from projects; developers are hired/fired;
  developers enter in vacation or take a day off; and developers have different
  experiences. This dynamic influences the assignment of \acp{cr}. Thus,
  contextual information is a necessity in automated approaches.
\end{enumerate}

\section{Overview of the Proposal}
\lipsum[1-5]