O gerenciamento eficiente de solicitações de mudança (SM) é fundamental para o
sucesso das atividades de manutenção e evolução de software. Entretanto, a
atribuição de SMs a desenvolvedores de software é um aspecto custoso desse
gerenciamento, pois demanda tempo e requer conhecimento apropriado do projeto de
software. Com o propósito de diminuir esse custo, várias pesquisas já propuseram
métodos de atribuição automática de SMs. Embora representem avanços na área,
existem vários fatores inerentes a atribuição de SMs que não são considerados
nessas pesquisas e são essenciais para a automação.

Como demonstrado nesse trabalho, a atribuição automática deve, por exemplo,
considerar a carga de trabalho, a experiência e o conhecimento dos
desenvolvedores, a prioridade e a severidade das SMs, a afinidade dos
desenvolvedores com os problemas descritos nas SMs, e até mesmo os
relacionamentos interpessoais. Para tornar esse cenário ainda mais complexo,
esses fatos podem variar de acordo com o projeto de software que está sendo
desenvolvido. Assim, uma solução para o problema de atribuição de SMs depende de
informações contextuais.

Assim, esse trabalho propõe, implementa e valida uma solução arquitetural
sensível ao contexto para atribuição automática de SMs. Dado o aspecto
contextual da solução, a arquitetura enfatiza a necessidade de considerar as
diversas fontes de informações presentes na organização, assim como a
necessidade de se desenvolver algorítimos que implementem diferentes estratégias
de atribuição. A proposta e implementação dessa solução é embasada em resultados
de duas pesquisas quantitativas: um estudo de mapeamento sistemático da
literatura, e uma pesquisa de questionário com desenvolvedores de software. Esse
último forneceu um conjunto de requisitos que a solução automatizada deve
satisfazer para que as estratégias de atribuição sejam atendidas, enquanto o
mapeamento da literatura identificou técnicas, algoritmos, e outros requisitos
necessários a automação.

A implementação da arquitetura segue uma estratégia de automação, também
elabo\-rada nesse trabalho, que possui dois componentes principais: um sistema
especialista baseado em regras (SEBR); e um modelo de recuperação de informação
(MRI) com técnicas de aprendizagem. Em nossa estratégia, esses dois componentes
são executados alternadamente em momentos diferentes a fim de atribuir uma SM
automaticamente. O SEBR processa regras simples e complexas, considerando
informações contextuais do projeto de software e da organização que o
desenvolve. O MRI é utilizado para fazer o casamento entre SMs e desenvolvedores
de acordo com o histórico de atribuições.

\begin{keywords}
Engenharia de Software, Manutenção e Evolução de Software, Gerenciamento de
Solicitações de Mudança, Atribuição Automática de Solicitações de Mudança
\end{keywords}